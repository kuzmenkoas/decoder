\documentclass[12pt,a4paper,openany]{article}
\usepackage[english,russian]{babel}
\usepackage{bstyle}

\begin{document}
\section{Описание}
Программа decoder декодирует результаты измерений программой Digitizer 1.1.1, сохраненные в бинарном формате. 
Программа позволяет сохранять результате в формате .root и .txt и гистограммы в формате .root и .png.

Исходный код программы находится в репозитории: \href{https://github.com/kuzmenkoas/decoder}{GitHub}.

\section{Зависимости}
\begin{itemize}
    \item ROOT
    \item cmake and make
    \item C++17
\end{itemize}

\section{Инструкция по установке}
\subsection{Linux}
Установка зависимостей (для Ubuntu/Linux Mint):
\begin{lstlisting}[language=bash]
sudo apt install gcc cmake
\end{lstlisting}

Установка программы:
\begin{lstlisting}[language=bash]
git clone https://github.com/kuzmenkoas/decoder
cd decoder
mkdir build
cd build
cmake ..
make
\end{lstlisting}

\subsection{Windows}
\begin{enumerate}
    \item Установить \href{https://visualstudio.microsoft.com/downloads/}{Visual Studio}.
    \item Установить \href{https://cmake.org/download/}{CMake}.
    \item Скомпилировать проект с помощью утилиты CMake (использовать компилятор MSVC).
    \item Собрать проект с помощью Visual Studio (кнопка decoder.exe).
    \item Добавить путь к исполняемому файлу (путь/decoder/out/build/x64-Debug/) в переменную окружения PATH (Win+R, вызвать sysdm.cpl).
\end{enumerate}

\section{Запуск}
Для запуска необходимо указать путь к исполняемому файлу (в Linux build/decoder), в Windows, если добавить путь в переменную окружения PATH, то из любой директории можно вызвать decoder.exe.

Программа позволяет использовать два варианта задания конфигурации декодирования (с помощью командной строки и с помощью конфигурационного файла).

Так же у программы предусмотрено различное поведение в зависимости от количества и типа входных бинарных файлов (PSD и Waveform).

Очередность передаваемых аргументов программе строго типизована. 
Первым, если нужен, передается конфигурационный файл, далее идут бинарные файлы, если передавать два бинарных файла, то сначала PSD, после Waveform.

Далее, для примера будут использованы конфигурационные файлы, которые лежат в директории decoder/cfg и бинарные файлы в директории decoder/data.

Декодировать PSD и Waveform с помощью конфигурационного файла:
\begin{lstlisting}[language=bash]
decoder.exe cfg/psdWaveform.cfg data/raw_data_psd_wavetest_2025_11_06__13_28_09.bin data/raw_waveform_psd_wavetest_2025_11_06__13_28_09.bin
\end{lstlisting}

Декодировать PSD и Waveform с помощью командной строки:
\begin{lstlisting}[language=bash]
decoder.exe data/raw_data_psd_wavetest_2025_11_06__13_28_09.bin data/raw_waveform_psd_wavetest_2025_11_06__13_28_09.bin
\end{lstlisting}

Декодировать PSD с помощью конфигурационного файла:
\begin{lstlisting}[language=bash]
decoder.exe cfg/psd.cfg data/raw_data_psd_wavetest_2025_11_06__13_28_09.bin
\end{lstlisting}

Декодировать PSD с помощью командной строки:
\begin{lstlisting}[language=bash]
decoder.exe data/raw_data_psd_wavetest_2025_11_06__13_28_09.bin
\end{lstlisting}

Декодировать Waveform с помощью конфигурационного файла:
\begin{lstlisting}[language=bash]
decoder.exe cfg/waveform.cfg data/raw_waveform_psd_wavetest_2025_11_06__13_28_09.bin
\end{lstlisting}

Декодировать Waveform с помощью командной строки:
\begin{lstlisting}[language=bash]
decoder.exe data/raw_waveform_psd_wavetest_2025_11_06__13_28_09.bin
\end{lstlisting}

\section{Конфигурация декодера}
Есть два парсера для конфигурации декодера -- через командную строку и через конфигурационный файл.

\subsection{Командная строка}
После запуска программы, в зависимости от количества бинарных файлов и их типов, различные конфигурационные модули запустятся.

\textbf{1. Тип бинарного файла}

Если передать 1 бинарный файл в программу, то выведется следующее сообщение:
\begin{lstlisting}
What file is it?
 (1) PSD
 (2) Waveform
\end{lstlisting}

Введите 1 для типа PSD и 2, если бинарный файл содержит форму сигнала (waveform).
Это влияет на алгоритм декодирования.

\textbf{2. Формат сохранения результатов}

Следующий вопрос появится вне зависимости от количества переданных файлов и их типов.
\begin{lstlisting}
Choose format to save
 (1) Root
 (2) Txt
 (3) Root and Txt
\end{lstlisting}

Необходимо ввести 1, 2 или 3 для выбора формата сохранения декодированных данных.

\textbf{3. Какие данные находятся в бинарном файле PSD}

Данный текст появится, если  есть входной бинарный файл PSD. 
Он говорит программе, какие данные сохранены в бинарном файле (необходимо для корректного декодирования). 
Зная параметры, программа в определенной последовательности декодирует данные (зная, сколько байт уходит на тот или иной параметр).

\begin{lstlisting}
What is encode in binary PSD file.
Choose everything that is stored in bin file 
(decoder will not be able to work correctly if this parameters will be incorrect)!
For multiple write as 123 - for qShort, qLong and cfd_y1. 
If in binary file stored all parameters choose 0 (ALL):
 (1) qShort
 (2) qLong
 (3) cfd_y1
 (4) cfd_y2
 (5) baseline
 (6) height
 (7) eventCounter
 (8) eventCounterPSD
 (9) psdValue
 (0) ALL
\end{lstlisting}

Введите цифры выбранных параметров непрерывно. 
Например, 126 для qShort, qLong и height, 157 для qShort, baseline и eventCounter.
Если в бинарном файле сохранены все параметры, то достаточно ввести только 0.

\textbf{4. Перевернуть сигнал или нет}

Этот модуль появится если:
\begin{enumerate}
    \item бинарный файл PSD содержит qShort, qLong или baseline.
    \item для переданного бинарного файла waveform модуль появится в любом случае.
    \item если есть оба бинарных файла -- модуль появится.
\end{enumerate}

\begin{lstlisting}
Reverse integral?
 (1) Yes
 (2) Not
\end{lstlisting}

Если введете 1, то результат baseline, qShort и qLong будет перевернут (умножен на -1).

\textbf{5. Конфигурация формы сигнала}

Модуль появится, если в входных файлах есть бинарный файл waveform.

\begin{lstlisting}
The number of baseline points?
\end{lstlisting}

\begin{lstlisting}
The number of qShort points?
\end{lstlisting}

\begin{lstlisting}
The number of qLong points?
\end{lstlisting}

Для всех ним необходимо, чтобы пользователь ввел целое число (integer).
Для одного события у формы сигнала есть определенная длина (настраивается в программе Digitizer).
Baseline points необходимо для определения шума (baseline), определяется как среднее значение в диапазоне (первые точки сигнала от 0 до введенного пользователем).
Далее, от baseline points до qShort points определяется значение интеграла qShort.
И последнее, от baseline points до qLong points определяется значение интеграла qLong.

Данное сообщение появится, если входным бинарным файлом будет только waveform.
\begin{lstlisting}
The number of waveform points?
\end{lstlisting}

Необходимо ввести целое число, характеризующую длину форму сигнала (задается в программе Digitizer, обязаны совпасть при декодировании, иначе программа некорректно сработает).

\textbf{6. Конфигурация гистограмм}

Данное сообщение появится независимо от входных файлов.
Здесь настраивается гистограмма по декодированным данным (для PSD и Waveform шаблон одинаковый, количество вариантов данных -- разное).
\begin{lstlisting}
Choose parameter to plot (multiply input, example: 123 for 3 parameters to plot)
 (0) exit
 (1) qShort
 (2) qLong
 (3) cfd_y1
 (4) cfd_y2
 (5) baseline
 (6) height
 (7) eventCounter
\end{lstlisting}

Необходимо непрерывно ввести все варианты, для которых необходима гистограмма. 
Если гистограмма не нужна, то достаточно ввести значение 0.

Далее, если необходима гистограмма для каждого параметра появятся выводы с настройкой.
\begin{lstlisting}
Configure a plot from PSD data for parameter qShort

Enter number of bins
\end{lstlisting}

Настраивается количество бинов -- целое число.
\begin{lstlisting}
Enter min value of histogram
\end{lstlisting}

Необходимо ввести минимальное значение для гистограммы (целое число, может быть отрицательным).
\begin{lstlisting}
Enter max value of histogram
\end{lstlisting}

Необходимо ввести максимальное значение для гистограммы (целое число, может быть отрицательным).

\subsection{Конфигурационный файл}
Можно все настройки для декодера написать в конфигурационном файле и скормить программе.

Структура файла:
\begin{lstlisting}
Output
Reverse
DataPSD
DataWaveform
WaveformConfig
Histogram
\end{lstlisting}

Для любого файла обязательными полями являются: Output.

Необязательные: Reverse, Histogram.

Для PSD бинарного файла необходимо заполнить поле DataPSD.

Для Waveform бинарного файла необходимо запонить поля DataWaveform и\\ 
WaveformConfig.

В Output настраивается формат вывода (txt и root): RootNtuple и TxtNtuple.

Выглядит как:
\begin{lstlisting}
Output
+ RootNtuple
+ TxtNtuple
\end{lstlisting}

В результате все выведется в root файл и txt файлы.

Если необходимо сохранить только в root файле (т.к. стоит учитывать, что вывод в txt формате долгий), то:
\begin{lstlisting}
Output
+ RootNtuple
\end{lstlisting}

Значения baseline, qShort и qLong (интегралы) можно реверсировать (умножить на -1). Делается с помощью параметра Reverse:
\begin{lstlisting}
Reverse true
\end{lstlisting}

или
\begin{lstlisting}
Reverse false
\end{lstlisting}

При значении true -- умножит эти значения (для PSD и Waveform данных) на -1, при значении false -- не будет умножать.

В DataPSD необходимо указать все переменные, которые закодированы в бинарном файле (если некорректно указать, то результат декодирования будет неверным).

Все возможные перменные указаны ниже:
\begin{lstlisting}
DataPSD
+ qShort
+ qLong
+ cfd_y1
+ cfd_y2
+ baseline
+ height
+ eventCounter
+ eventCounterPSD
+ psdValue
\end{lstlisting}

В DataWaveform указывается, какие переменные необходимо получить. Есть 4 параметра:
\begin{lstlisting}
DataWaveform
+ qShort
+ qLong
+ baseline
+ entries
\end{lstlisting}

Из нового тут только параметр $entries$ -- сохраняет в отдельном ntuple значения формы сигнала как по одному событию.
Есть значения $id$ -- номер события, $t$ -- момент времени, когда получено значение сигнала, $wave$ -- значение сигнала.

Далее, в WaveformConfig задаются значения для декодирования: точки отсчета для baseline, qShort и qLong (как в программе Digitizer, при этом необязательно задавать такие же) и длина сигнала (wavelength).
\begin{lstlisting}
WaveformConfig
+ baseline 100
+ qShort 120
+ qLong 380
+ wavelength 1000
\end{lstlisting}

При этом, значение wavelength обязано быть таким же, как при измерениях, и его необязательно указывать, если программа получает два входных бинарных файла -- PSD и Waveform.

Для настройки гистограмм необходимо указать: тип файла, параметр, количество бинов, минимальное значение, максимальное значение. Именно в такой последовательности!

Пример:
\begin{lstlisting}
Histogram
+ PSD qShort 1000 -20000 200000
+ Waveform qShort 1000 -20000 200000
\end{lstlisting}

Пример конфигурационного файла для декодирования PSD и Waveform:
\begin{lstlisting}
Output
+ RootNtuple
+ TxtNtuple

Reverse false

DataPSD
+ qShort
+ qLong
+ cfd_y1
+ cfd_y2
+ baseline
+ height
+ eventCounter
+ eventCounterPSD
+ psdValue

DataWaveform
+ qShort
+ qLong
+ baseline
+ entries

WaveformConfig
+ baseline 100
+ qShort 120
+ qLong 380

Histogram
+ PSD qShort 1000 -20000 200000
+ Waveform qShort 1000 -20000 200000
\end{lstlisting}

Пример конфигурационного файла для декодирования только PSD:
\begin{lstlisting}
Output
+ RootNtuple
+ TxtNtuple

Reverse false

DataPSD
+ qShort
+ qLong
+ cfd_y1
+ cfd_y2
+ baseline
+ height
+ eventCounter
+ eventCounterPSD
+ psdValue

Histogram
+ PSD qShort 1000 -20000 200000
\end{lstlisting}

Пример конфигурационного файла для декодирования только Waveform:
\begin{lstlisting}
Output
+ RootNtuple

Reverse false

DataWaveform
+ qShort
+ qLong
+ baseline
+ entries

WaveformConfig
+ baseline 100
+ qShort 120
+ qLong 380
+ wavelength 1000

Histogram
+ Waveform qShort 1000 -20000 200000
\end{lstlisting}

\end{document}